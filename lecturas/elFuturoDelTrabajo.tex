\subsection{El futuro del trabajo Ámerica latina y el Caribe}
\subsubsection{En pocas palabras}
\subsubsection{Por qué este tema}
\subsubsection{Que esta pasando}
Las dos grandes tendencias que tiene latinoamerica para el futuro del trabajo son {\bf el cambio tecnológico} y el {\bf envejecimiento poblacional}. Ambas se suman a los efectos de la globalización de bienes y servicios, lo posbile cambios del cambio climatico y muchos mas problemas aun no determinados

\begingroup
\setlength{\tabcolsep}{12pt} % Default value: 6pt
\renewcommand{\arraystretch}{1.5} % Default value: 1
\begin{tabular}{p{4cm}|p{11cm}}
\multicolumn{2}{c}{\large \bf Las grandes tendencias: tecnologia y demografía}  \\
Tecnología y demografía son dos tendencias radicalmente diferentes&Ambas tendencias provocaŕan cambios profundos en la manera de trabajar y organizarse en sociedades. Sin embargo ambas son muy diferentes debido a que: En el caso de la tecnologia esta es de {\bf impacto mediático} y {\bf sucede al instante}. En cambio, el evenjecimiento ocurre de {\bf manera gradual} y {\bf avanza más rápido de lo normal}\\ 
\end{tabular}
\endgroup


\begingroup
\setlength{\tabcolsep}{12pt} % Default value: 6pt
\renewcommand{\arraystretch}{1.5} % Default value: 1
\begin{tabular}{p{4cm}|p{11cm}}
\multicolumn{2}{c}{\large \bf  La tecnología como agente disruptivo}  \\
¿Por qué tatno énfasis en discutir el futuro del trabajo ahora?&Algunos creen que los avances tecnológicos actuales no se comparan al impacto de los de la revocion industrial en términos de crecimiento ecocómico y bienestar. Sin embargo, otros advierten que el cambio tecnológico va causar grandes impactos, aunque la tecnología por si sola no es suficiente y tenemos que verla como una aliada\\
El pasado muestra que los cambios tecnológicos impactan fuertemente el mercado&Como ejemplo en el sigo XIX decendio el 90\% de los trabajos agrícolas en USA. Y otro ejemplo es con la maquina de escribir que en 3 decadas decendio la mitad de trabajos en los sectores de manufactura\\
La diferencia a la cuarta revolución industrial de las anteriores es la velocidad de los cambios&Las alteraciones radicales en el mercado se producen demaciado rápido. Por ejemplo, cada año se duplica la potencia de los microchip y en dos décadas desde que se empezo a comercionalizar los primeros smartphone la mitad de la población tiene ahora uno.\\
Los cambios tecnológicos ventiginosos pueden causar problemas&Como nuestra capacidad de adaptación es limitada, tiende a complicar tanto a personas como gobiernos a explorar nuevas tecnologías. Esto se le puede llamar un {\bf tsunami tecnológico} porque lo que podia tardar varias generaciones, hot ocurre en pocos años\\
Existen barreras importantes que hacen difícil que América  Latina y el Caribe puedan absorber rápido este tsunami tecnológico&La región no cuenta con las capadiades e infractura necesaria para dar una cabida plena a esta revolución tenológica. (1) Los niveles de preparación de la mano de obra en la región son un problema para adoptar nuevas tecnologias. (2) El menor costo de mano de obra no es incentivo para innovar\footnote{Es exactamente lo que paso en la RI en reino unido}. (3) La mayoría de las firmas en la región son pequeñas y eso restringe la innovacion. (3) Falta financiamiento y capacidad técnica para llevar a cabo una transformación digital y finalmente (4) Falta mucha infractura, por ejemplo banda ancha\\
\end{tabular}
\endgroup


\begingroup
\setlength{\tabcolsep}{12pt} % Default value: 6pt
\renewcommand{\arraystretch}{1.5} % Default value: 1
\begin{tabular}{p{4cm}|p{11cm}}
\multicolumn{2}{c}{\large \bf  La demografía, lenta pero segura} \\
El mundo envejece y América Latina y el Caribe sucede más rápido que otras regiones&Durante el siglo XIX y XX se puede hablar de una explosión demográfica\footnote{Debido en gran parte a los avances médicos y de salubridad de las cuidades}. Sin embargo ahora nos encontramos en el fin del bono demográfico, eso quiere decir que la región enbejecerá más rapidamente hasta duplicar las cifras actuales\\
La región no solo verá un incremento del número de adultos mayores: también crecerá el porcentaje de personas que alcanzan la 'cuarta edad'.&Las personas llegaran más alla de la tercera edad, por lo que requerira cuidados adicionales. Se reducira la natalidad y hara que el tamaño de las familais se reduzca. Esto probocara que la vida activa se prolongue y tengan que trabajar durante más años. Los sistemas de pension y seguridad social tienen que cambiar, ya que por ahora s\\
\end{tabular}
\endgroup

\begingroup
\setlength{\tabcolsep}{12pt} % Default value: 6pt
\renewcommand{\arraystretch}{1.5} % Default value: 1
\begin{tabular}{p{4cm}|p{11cm}}
\multicolumn{2}{c}{\large \bf  Impactos en el futuro del trabajo}\\
La cuarta revolución industrial es una oportunidad que la región no pueden dejar pasar. &
La tecnología tiene la gran promesa de que todos los avances tecnológicos incrementaran la productividad de las economías, aunque esto solo pasara si se adoptan las más prometedoras. Y en terminos practicos podemos identificar dos principalmente las {\bf tecnologías de automatización}\footnote{Aquellas tecnologías que permiten automatizar tareas realizadas por seres humanos} y las {\bf tecnologías de intermediación}\footnote{Aquellas que aumentan la capacodad de conectar la oferta y la demanda}
\end{tabular}
\endgroup

\begingroup
\setlength{\tabcolsep}{12pt} % Default value: 6pt
\renewcommand{\arraystretch}{1.5} % Default value: 1
\begin{tabular}{p{4cm}|p{11cm}}
\multicolumn{2}{c}{\large \bf  ¿Destruirá la automatización nuestros trabajos?}\\
Existen un amplio debate sobre efecto de la automatización en el mercado de trabajo&Se le hace mucho eco de los potenciales efectos destructores de las tecnologías de automatización. Pero una cosa es el potencial desde el punto de vista tecnológico y otra es que tenga sentido desde el punto de vista economico. Dado el costo de contratación de un trabajador en muchos países de nuestra región es bajo, no resulta viable remplazar (aun) trabajadores por robot. Lo más importante es que {\bf el empleo no desaparecera, sino que se transformara}\\
No sabemos con exactitud qué sucederá en el siglo XXI, pero hay motivos para pensar que los seres humanos seguiremos teniendo trabajo&Existen estudios que dan por hecho que la tecnología tiene potencial para destruir ocupaciones completas, lo cual es poco probable y no hay evidencia empirica. La mayoria de ocupaciones tiene un número de tareas que pueden automatizarse, pero son pocas las que son por completo. Ademas las nuevas tecnologías traen consigo nuevas ocupaciones\footnote{Todas impulsadas en gran medidad por el auje de la inteligencia artificial}\\
\end{tabular}
\endgroup

\begingroup
\setlength{\tabcolsep}{12pt} % Default value: 6pt
\renewcommand{\arraystretch}{1.5} % Default value: 1
\begin{tabular}{p{4cm}|p{11cm}}
\multicolumn{2}{c}{\large \bf ¿Aumentará la desigualdad?}\\
L.A. y el Caribe se ha caracterizado durante décadas por ser la región más desigual del mundo& A pensar de que hemos avanzado en la reducción de la desigualdad, hay una probabilidad de que esta se vea amenazada por la cuarta revolución industrial\\    
\end{tabular}
\endgroup

\begingroup
\setlength{\tabcolsep}{12pt} % Default value: 6pt
\renewcommand{\arraystretch}{1.5} % Default value: 1
\begin{tabular}{p{4cm}|p{11cm}}
\multicolumn{2}{c}{\large \bf ¿Qué cambios traerán las tecnologías de intermediación?}\\
El surgimiento de plataformas digitales& Su mayor aporte sera juntar a oferentes con demandantes de servicios, reduciendo de manera radical los costos de transacción. Lo que de alguna manera a largo plazo hace aumentar la cantidad de trabajo y el capital efectivo de la economía, ya que se optimizan los recuersos que\\
¿Cuál es el impacto de estas tecnologías sobre las personas?&Se eliminaran barreras de acceso al trabajo y se podra optar a altos grados de flexibilidad de horarios. Esto podria facilitar la labor de adultos mayores o estudiantes a trabajos de tiempo parcial. Tambien facilita a las empresas a contratar personal sin crear una relación laboral. Todo esto favorece {\bf el auge de la economía por demanda}, ademas pone retos muy fuertes a los mecanismo de seguridad social.\\
\end{tabular}
\endgroup

\begingroup
\setlength{\tabcolsep}{12pt} % Default value: 6pt
\renewcommand{\arraystretch}{1.5} % Default value: 1
\begin{tabular}{p{4cm}|p{11cm}}
\multicolumn{2}{c}{\large \bf ¿Cual sera el impacto del envejecimiento?}\\
Una sociedad más envejecida crece menos&Hay menos trabajadores potenciales por cada ciudadano, lo que induce a una relentización del crecimiento económico, esto generara mas presión en los costos de salud y pensiones publicas. Tambien menores tasas de fertilidad generan más trabajo para mujeres y familias más pequeñas generan mayores ahorros. Otra opcion es que la escazes de trabajadores impulsara que la produccion se encamine más a las tecnologias. Esto en el marco que 1 de 4 personas en el 2010 sera mayor de 60 años. Finalmente esto generara un alza absurda de servicios médicos, de cuidados y de atención personal a personas mayores\\
¿Qué pasará en América Latina y el Caribe?& Hay que tener en cuenta que lo previamente dicho nos da una pista de hacia donde se estan moviendo los trabajos. El futuro traera más trabajos, pero la demanda de trabajos se movera\\

\end{tabular}
\endgroup





\subsubsection{Qué hay de nuevo}

\begingroup
\setlength{\tabcolsep}{12pt} % Default value: 6pt
\renewcommand{\arraystretch}{1.5} % Default value: 1
\begin{tabular}{p{4cm}|p{11cm}}
¿Ya se observan algunos efectos de la cuarta revolución industrial en América Latina y el Caribe?&\\
El empleo total creció de forma paulatina&\\
Ahora contamos con un nuevo aliado, los datos masivos&\\
\end{tabular}
\endgroup


\begingroup
\setlength{\tabcolsep}{12pt} % Default value: 6pt
\renewcommand{\arraystretch}{1.5} % Default value: 1
\begin{tabular}{p{4cm}|p{11cm}}
\multicolumn{2}{c}{\large \bf ¿Cómo enfrentar los desafíos?}\\
A medida que el diagnóstico se va completando, se pone en evidencia la necesidad de actuar rápido&\\
El Estado deberá acometer grandes transformaciones que surgen del cambio tecnológico y de la demografía&\\
Las empresas también deben acometer una transformación importante&\\
Finalmente, los individuos debemos aprender todo el tiempo&\\
Contamos con suficientes señales que garantizan que el futuro del traba, de una u otra forma, nos pondra a prueba&\\
\end{tabular}
\endgroup

\subsubsection{Qué sigue}