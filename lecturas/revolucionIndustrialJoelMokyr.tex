\section{Lecturas}
\subsection{La revolición industrial y la nueva historia - Joel Mokyr}

\begingroup
\setlength{\tabcolsep}{12pt} % Default value: 6pt
\renewcommand{\arraystretch}{1.5} % Default value: 1
\begin{tabular}{p{3cm}|p{12cm}}
  \multicolumn{2}{c}{AGRICULTURA, COMERCIO EXTERIOR, CRECIMIENTO DE LA población Y TECNOLOGIA}                                                                                                             \\ \hline
  Agricultura y transporte     & $\bullet$ La capacidad de la economía para generar alimentos sufientes fue una de las prioridades dentro de la epoca                                                      \\
                               & $\bullet$La producción y la distribución de alimentos usaba factores de producción no agricolás                                                                           \\
                               & $\bullet$ Gran parte de la revolición agricola se debe a la revolición del transporte                                                                                     \\
                               & $\bullet$ La produccion y la distribucion de alimentos usaba factores de produccion no agricolás                                                                          \\
                               & $\bullet$ La revolición Agricola fue en parte causada por la revolición industrial                                                                                        \\
                               & $\bullet$El transporte en Reino unido era mucho mejor que en Francia lo que ayudo bastante                                                                                \\ \hline
  Comercio  Exterior e Imperio & $\bullet$ Existen 2 hechos que hacen considerar importante el factor del comercio exterior tal como:                                                                      \\
                               & \hspace{1cm} $\bullet$ Se empieza a tener capacidad exportadora                                                                                                           \\
                               & \hspace{1cm} $\bullet$ Se empieza a importar determinados bienes que no se podían producir efecientemente o no habían\footnote{Como por ejemplo el algodon desde Estados} \\
                               & $\bullet$ La demanda externa proporciono los mercados necesarios para la produccion industrial                                                                            \\
                               & $\bullet$ Mecados exteriores con demanda $<$ Aumento de las ganancias                                                                                                     \\
                               & $\bullet$ En teoría las exportaciones "No fueron buenas para Gran Bretaña"                                                                                                \\
                               & $\bullet$ {\bf El comercio es hijo de la industria}                                                                                                                       \\
                               & $\bullet$ Que haya crecido el mercado exterior no significa que sea el culpable                                                                                           \\ \hline
  Cambio    técnico            & $\bullet$ Aumento en la productividad de los procesos                                                                                                                     \\
                               & $\bullet$ Se sustenta bajo 2 supuestos                                                                                                                                    \\
                               & \hspace{1cm} $\bullet$ Algunas actividades son complementos perfectos                                                                                                     \\
                               & \hspace{1cm} $\bullet$ Existen costos al hacer cambios radicales en la asignacion de recursos                                                                             \\                                                                                  \\
                               & $\bullet$ Las invenciones creaban cuellos de botella que se trataban de solucionar lo más rapido posible                                                                  \\
                               & $\bullet$ Que subiera el precio de la madera y que por eso se empezara a utilizar carbon no es un argumento para decir que es una razon para la revolición industrial
\end{tabular}
\endgroup

\begingroup
\setlength{\tabcolsep}{12pt} % Default value: 6pt
\renewcommand{\arraystretch}{1.5} % Default value: 1
\begin{tabular}{p{3cm}|p{11cm}}
  \multicolumn{2}{c}{\large LOS FACTORES DE PRODUCCION: TRABAJO Y CAPITAL}                                                                                          \\ \hline
  Trabajo & $\bullet$ Se puede es trabajar desde 2 enfoques:                                                                                                        \\
          & \hspace{1cm} $\bullet$ Recurso escaso: Por lo que en los lugares donde fuera abundante y barato tendrá exito la revolición Industrial                   \\
          & \hspace{1cm} $\bullet$ Ahorrar trabajo y su escases proboco un avance de lo técnico                                                                     \\
          & $\bullet$ Es más llamativo invertir en ser técnico que en aumentar los salarios                                                                         \\
          & $\bullet$ Los salarios más bajos ayudan al aumento del capital de manera más veloz                                                                      \\
          & $\bullet$ Eficiencia del trabajo dependia de educacion y nutricion siendo la ultima la más importante                                                   \\
          & $\bullet$ La gente se iba a las fabricas por "Mejores salarios" y por el miedo a que sus trabajos desaparecieran                                        \\
          & $\bullet$ La industria domestica {\bf NO} fue una condicion necesaria ni suficiente para el alza de la industria                                        \\
          & $\bullet$ No es el traspaso sino la tranformacion                                                                                                       \\
          & $\bullet$ La caida de la industria en Irlanda y quedó como mano de obra barata                                                                          \\\hline
  Capital & $\bullet$ La taza de inversión se duplicó en la Revolición Industrial                               ´                                                   \\
          & $\bullet$ Paso de un capital circulante a un capital fijo                                                                                               \\
          & $\bullet$ La acumulación de capital no fue tan decisivo                                                                                                 \\
          & $\bullet$ La necesidad de capital se solventa como:                                                                                                     \\
          & \hspace{1cm} $\bullet$ Utilizar la riqueza privada para empezar y reinvertir los beneficios en la empresa                                               \\
          & \hspace{1cm} $\bullet$ Conseguir dinero de familiares y amigos                                                                                          \\
          & \hspace{1cm} $\bullet$ Existían entidades financieras                                                                                                   \\
          & $\bullet$ La escasez de capital y los sesgos de mercado fueron posible factores en el descenso de la taza de acumulación y la limitada movilidad social \\
\end{tabular}
\endgroup




\begingroup
\setlength{\tabcolsep}{12pt} % Default value: 6pt
\renewcommand{\arraystretch}{1.5} % Default value: 1
\begin{tabular}{p{15cm}}
  \multicolumn{1}{c}{\large LAS CONSECUENCIAS: EL DEBATE SOBRE EL NIVEL DE VIDA}                                                                             \\ \hline
  $\bullet$ Entre 1760 y 1830 susedieron 4 grandes acontecimientos que cambiaron Gran Bretaña                                                                \\
  \hspace{1cm} $\bullet$ Las series de malas cosechas que se dieron entre mediados del siglo XVIII y el final de las guerras napoleonicas                    \\
  \hspace{1cm} $\bullet$ El estado de guerra que mantubo Gran Bretaña                                                                                        \\
  \hspace{1cm} $\bullet$ La aceleración del crecimiento de la población en el siglo XVIII                                                                    \\
  \hspace{1cm} $\bullet$ La acumulación de: Cambio técnico, acumulación de capital, los deplazamientos sectoriales y los cambios de la revolición Industrial \\
  $\bullet$ Disminuyó el nivel de vida por culpa de las malas cosechas y aumentaron el precio de los alimentos para la población en general                  \\
  $\bullet$ Las malas cosechas aumentaron el precio de la comida y la renta                                                                                  \\
  $\bullet$ Mayores impuesto que se aplicaron a la población                                                                                                 \\
  $\bullet$ Crece la población y el nivel de vida de la población baja considerablemente                                                                     \\
  $\bullet$ Aumento de la desigualdad en la población                                                                                                        \\
  $\bullet$ Hay demasiados vacíos de datos como para dar cifras y sacar concluciones de esos tiempo                                                          \\
  $\bullet$ Los desequilibrios crearon oportunidades sin precedentes para conseguir grandes beneficios                                                       \\
  $\bullet$ El caso contrafactural, como entender y tener una mejor idea de los datos:                                                                       \\
  \hspace{1cm} $\bullet$Estimando aproximadamente los efectos de las otras fuerzas                                                                           \\
  \hspace{1cm} $\bullet$Examinar otras economías en ese mismo tiempo                                                                                         \\
\end{tabular}
\endgroup


\newpage